\documentclass[12pt]{extreport}
\usepackage[a4paper]{geometry}
\usepackage{extsizes}
\usepackage{amsthm}
\usepackage{amsmath}
\usepackage{amssymb}
\newtheorem*{thm}{Theorem}

\begin{document}
\title{MAT 180 Term Paper}
\author{Darya Chumakova and Satoshi Scott Iwako}
\maketitle

\begin{thm}
For any $n$ positive numbers $a_1, a_2, ... a_n$: 
\begin{align*}
\frac{n}{1/a_1 + 1/a_2 + ... + 1/a_n} \leq \sqrt[n]{a_1 a_2 a_3 ... a_n} \leq \frac{a_1 + a_2 + ... + a_n}{n}
\end{align*}
Where the equality holds iff $a_1 = a_2 = ... = a_n$.
\end{thm}

\begin{proof}
We will first prove:
\begin{align*}
\sqrt[n]{a_1 a_2 a_3 ... a_n} \leq \frac{a_1 + a_2 + ... + a_n}{n}
\end{align*}

\begin{flushleft}
\textit{Forward Induction:} (To show statement holds for $n = 2^k$ where $k = 1, 2, 3...$) \\
\textbf{Base Case:} We wish to show the inequality holds for n = 2 i.e. $$\sqrt{ab} \leq \frac{a+b}{2}$$
Suppose $a$ and $b$ are positive numbers. Then $(a-b)^2 \geq 0$. From here it follows that:
\begin{align*}
(a-b)^2 \geq 0 &\iff a^2 - 2ab + b^2 \geq 0 \\
			   &\iff a^2 + 2ab + b^2 \geq 4ab \\
			   &\iff (a+b)^2 \geq 4ab \\
			   &\iff (a+b) \geq 2\sqrt{ab} \\
			   &\iff \frac{a+b}{2} \geq \sqrt{ab}
\end{align*}
Hence the inequality holds for $n=2$.\\
\textbf{Inductive Step:} Suppose $$\sqrt[n]{a_1 a_2 a_3 ... a_n} \leq \frac{a_1 + a_2 + ... + a_n}{n} \qquad (1)$$ holds for $n = 2^k$. We wish to show the inequality holds for $n = 2^{k+1} = 2 \cdot 2^k = 2n$. Let:
$$a_1 = \frac{b_1 + b_2}{2} \qquad a_2 = \frac{b_3 + b_4}{2} \qquad \ldots \qquad a_n = \frac{b_{2n-1} + b_{2n}}{2}$$
where $b_1, b_2 ... b_{2n}$ are positive numbers. Next, we will substitute the above values into $(1)$. 
\begin{align*}
\frac{ (\frac{b_1 + b_2}{2} + \frac{b_3 + b_4}{2} + \ldots + \frac{b_{2n-1} + b_{2n}}{2}) } {n} \geq \sqrt[n]{(\frac{b_1 + b_2}{2})\cdot(\frac{b_3 + b_4}{2})\cdots (\frac{b_{2n-1} + b_{2n}}{2})} &\iff
\frac{b_1 + b_2 + b_3 + b_4 + \ldots + b_{2n-1} + b_{2n}}{2n}
%\geq \sqrt[n]{(\frac{b_1 + b_2}{2})\cdot(\frac{b_3 + b_4}{2})\cdots (\frac{b_{2n-1} + b_{2n}}{2})}
\end{align*}

\end{flushleft}
\end{proof}

% Satoshi's Backward Induction goes here:
\begin{flushleft}
\textit{Backwards Induction}
\begin{proof}
Now that we have established that the geometric and arithmetic ineqality holds for $n = 2^k$ we need to use a non-standard backwards induction to show that the inequality holds for all $n \in \mathbb{N}$.
\newline
\newline
$\boxed{\text{Base Case}}$ What we want to do is to assume that the following works for $n = 2^k$ and then prove that it works for $n - 1$. We can assume that the inequality holds for $n = 4$ lets prove that the inequality holds for $n = 3$. $$\sqrt[4]{a_1a_2a_3a_4} \leq \frac{a_1 + a_2 + a_3 + a_4}{4}$$ We apply a special trick to the inductive hypothesis by relabeling some of the $a$'s as follows: Let $a_1 = b_1, a_2 = b_1, a_3 = b_3$. Now we get the following inequality: $$\sqrt[4]{b_1b_2b_3a_4} \leq\frac{b_1 + b_2 + b_3 + a_4}{4}$$ Now from here we want to find an $a_4$ such that,

\begin{align*}
\frac{b_1 + b_2 + b_3 + a_4}{4} = \frac{b_1 + b_2 + b_3}{3} &\iff a_4 = \frac{4}{3}(b_1 + b_2 + b_3) - (b_1 + b_2 + b_3) \\&\iff a_4 = 	\frac{b_1 + b_2 + b_3}{3}.
\end{align*}
So, let $a_4 = (b_1 + b_2 + b_3)/3$ then we can rewrite the inequality as follows:
\begin{align*}
\sqrt[4]{b_1b_2b_3(\frac{b_1 + b_2 + b_3}{3})} \leq \frac{b_1 + b_2 + b_3}{3} &\iff b_1b_2b_3(\frac{b_1 + b_2 + b_3}{3}) \leq (\frac{b_1 + b_2 + b_3}{3})^4 \\&\iff b_1b_2b_3 \leq (\frac{b_1 + b_2 + b_3}{3})^3 \\&\iff \sqrt[3]{b_1b_2b_3} \leq \frac{b_1 + b_2 + b_3}{3}.
\end{align*}
Hence the base case holds. Now assume that the inequality holds for $n = 2^k$. Then, $$\sqrt[n]{a_1a_2\cdots a_n} \leq \frac{a_1 + a_2 + \ldots + a_n}{n} $$ We want to show that the inequality holds for $n-1$. Using the same trick again:
\begin{align*}
a_1 = b_1, \ a_2 = b_2, \ a_3 = b_3,\ldots, a_{n-1} = b_{n-1}
\end{align*}
Replacing the $a_i$ with its respective $b_i$ for $i = 1,\ldots, n-1$ we get:
\begin{align*}
\sqrt[n]{b_1b_2\cdots b_{n-1}a_n} \leq \frac{b_1 + b_2 + \ldots + b_{n-1} + a_n}{n}
\end{align*}
Again we want to find an $a_n$ such that:
\begin{align*}
\frac{b_1 + b_2 + \ldots + b_{n-1} + a_n}{n} = \frac{b_1 + b_2 + \ldots + b_{n-1}}{n-1}
\end{align*}
Solving for $a_n$ we get that:
\begin{align*}
a_n = \frac{b_1 + b_2 + \ldots + b_{n-1}}{n-1}
\end{align*}
Using the $a_n$ above we get that:
\begin{align*}
\sqrt[n]{b_1b_2\cdots b_{n-1} \big(\frac{b_1 + b_2 + \ldots + b_{n-1}}{n-1}\big)} \leq \frac{b_1 + b_2 + \ldots + b_{n-1}}{n-1} &\iff\\
b_1b_2\cdots b_{n-1} \big(\frac{b_1 + b_2 + \ldots + b_{n-1}}{n-1}\big) \leq \big(\frac{b_1 + b_2 + \ldots + b_{n-1}}{n-1}\big)^n &\iff\\
b_1b_2\cdots b_{n-1} \leq \big(\frac{b_1 + b_2 + \ldots + b_{n-1}}{n-1}\big)^{n-1} &\iff\\
\sqrt[n-1]{b_1b_2\cdots b_{n-1}} \leq \big(\frac{b_1 + b_2 + \ldots + b_{n-1}}{n-1}\big)
\end{align*}
And thus, achieving the desired result.
\end{proof}
\end{flushleft}
\end{document}
